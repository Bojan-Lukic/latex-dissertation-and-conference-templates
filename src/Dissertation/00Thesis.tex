\documentclass[12pt, a4paper, bibtotoc, numbers=noenddot]{scrreprt}


\usepackage[sorting=none]{biblatex}
\usepackage{xr}
\externaldocument{13Chapter3}
\usepackage[main=british,ngerman]{babel}
\usepackage[onehalfspacing]{setspace}
\usepackage[utf8]{inputenc}
\usepackage{textcomp}
\usepackage{listings, lstautogobble}
%\lstset{language=Python, basicstyle=\footnotesize, autogobble=true,}
\lstdefinestyle{mystyle}{
	literate=*{=}{{\bfseries\textcolor{codepurple}{=}}}{1}
	{==}{{\bfseries\textcolor{codepurple}{==}}}{2}
	{*}{{\bfseries\textcolor{codepurple}{*}}}{1}
	{+}{{\bfseries\textcolor{codepurple}{+}}}{1}
	{<}{{\bfseries\textcolor{codepurple}{<}}}{1}
	{>}{{\bfseries\textcolor{codepurple}{>}}}{1}
	{>=}{{\bfseries\textcolor{codepurple}{>=}}}{2}
	{<=}{{\bfseries\textcolor{codepurple}{<=}}}{2}
	{True}{{\bfseries\textcolor{codegreen}{True}}}{4}
	{False}{{\bfseries\textcolor{codegreen}{False}}}{5}
	{1}{{\textcolor{codegreen}{1}}}{1},
	commentstyle=\color{teal},
	keywordstyle=\bfseries\color{codegreen},
	numberstyle=\tiny\color{codegray},
	stringstyle=\color{red},
	basicstyle=\ttfamily\footnotesize,
	breakatwhitespace=false,         
	breaklines=true,                 
	captionpos=b,                    
	keepspaces=true, 
	showspaces=false,                
	showstringspaces=false,
	showtabs=false,                  
	tabsize=2,
	numbers=left,
	numbersep=5pt, 
	xleftmargin=2em,
	framexleftmargin=1.5em,
	escapeinside=||
}
\lstset{style=mystyle}
\let\origthelstnumber\thelstnumber
\makeatletter
\newcommand*\Suppressnumber{%
	\lst@AddToHook{OnNewLine}{%
		\let\thelstnumber\relax%
		\advance\c@lstnumber-\@ne\relax%
	}%
}
\newcommand*\Reactivatenumber[1]{%
	\setcounter{lstnumber}{\numexpr#1-1\relax}
	\lst@AddToHook{OnNewLine}{%
		\let\thelstnumber\origthelstnumber%
		\refstepcounter{lstnumber}
	}%
}
\makeatother

\usepackage{notoccite}
\usepackage[usenames,dvipsnames,svgnames,table]{xcolor}
\definecolor{codegreen}{rgb}{0,0.6,0}
\definecolor{codegray}{rgb}{0.5,0.5,0.5}
\definecolor{codepurple}{rgb}{0.58,0,0.82}
\addtokomafont{disposition}{\rmfamily}
\setlength{\emergencystretch}{1em}
\PassOptionsToPackage{hyphens}{url}

\usepackage{svg}
\usepackage{pdflscape}
\usepackage[toc,page]{appendix}
\usepackage{hyphenat}
\usepackage[style=english]{csquotes}
\usepackage{totpages}

\usepackage{datetime}
\newdateformat{britishdate}{\ordinal{DAY} \monthname[\THEMONTH] \THEYEAR}
\newdateformat{germandate}{\THEDAY.\,\monthname[\THEMONTH] \THEYEAR}

\usepackage[format=plain, justification=centering]{caption}
\usepackage[margin=10pt,font=small,labelfont=bf]{caption}
\usepackage{array,longtable,multirow}

\usepackage{tabularx}
\renewcommand\tabularxcolumn[1]{m{#1}}
\newcommand{\leftrighttext}[3][10em]{% 
	\begin{tabularx}{\linewidth}[t]{@{}Xp{#1}@{}} 
		#2 & #3 
	\end{tabularx}% 
}

\usepackage{booktabs}
\usepackage{mathtools,mathcomp,amsmath,amssymb,amsfonts,array}
\usepackage{upgreek}
\usepackage{extarrows}
\usepackage[shortcuts]{extdash}
\usepackage{xfrac}

\usepackage{nicefrac,siunitx,units}
\DeclareSIUnit\Darcy{D}
\DeclareSIUnit\wtpercent{\textsubscript{wt}\%}
\DeclareSIUnit\volpercent{\textsubscript{vol}\%}
\DeclareSIUnit\molpercent{\textsubscript{mol}\%}

\usepackage{graphics, graphicx, graphbox}
\usepackage{tikz}
\usepackage{blindtext,wrapfig}
\makeatletter
\newcommand\wrapfill{\par
	\ifx\parshape\WF@fudgeparshape
	\nobreak
	\vskip-\baselineskip
	\vskip\c@WF@wrappedlines\baselineskip
	\allowbreak
	\WFclear
	\fi
}
\makeatother 
\usepackage{fancyhdr}

%% Control of length head
\newlength\FHoffset
\setlength\FHoffset{-0cm}
\addtolength\headwidth{2\FHoffset}
\fancyheadoffset{\FHoffset}

%% Control of length foot
\newlength\FFoffset
\setlength\FFoffset{-0cm}
\addtolength\headwidth{2\FFoffset}
\fancyfootoffset{\FFoffset}

%% Control of with and colour of head/foot
\renewcommand{\headrulewidth}{1pt}
\renewcommand{\footrulewidth}{1pt}
\renewcommand{\headrule}{\hbox to\headwidth{\color{gray}\leaders\hrule height \headrulewidth\hfill}}
\renewcommand{\footrule}{\hbox to\headwidth{\color{gray}\leaders\hrule height \footrulewidth\hfill}}

%% Set different pagestyles
\fancypagestyle{plain}{%
	\fancyhf{} % clear all header and footer fields
	\fancyfoot[R]{\thepage} % except the center
	\renewcommand{\headrulewidth}{0pt}
	\renewcommand{\footrulewidth}{0pt}}
%
\fancypagestyle{new}{
\fancyhf{} %clears plain style
\addtolength{\headheight}{\baselineskip}
\fancyhead[R]{\leftmark}
\fancyfoot[R]{\thepage}
\renewcommand{\headrulewidth}{1pt}
\renewcommand{\footrulewidth}{0pt}%
%define \leftmark style, add another "#" for debugging
%\renewcommand{\chaptermark}[1]{\markboth{##1}{}}
\renewcommand{\chaptermark}[1]{%
	\markboth{\MakeUppercase{%
			\chaptername}\ \thechapter.%
		\ ##1}{}}
}
%
\fancypagestyle{appendix}{
	\fancyhf{} %clears plain style
	\addtolength{\headheight}{\baselineskip}
	\fancyhead[R]{\appendixname}
	\fancyfoot[R]{\thepage}
	\renewcommand{\headrulewidth}{1pt}
	\renewcommand{\footrulewidth}{0pt}%
	%define \leftmark style, add another "#" for debugging
	%\renewcommand{\chaptermark}[1]{\markboth{##1}{}}
	\renewcommand{\chaptermark}[1]{%
		\markboth{\MakeUppercase{%
				\chaptername}\ \thechapter.%
			\ ##1}{}}
}
%
\renewcommand*\chapterpagestyle{plain} 


%%%%% Fonts %%%%%
%% With XeLaTeX and the following option it is possible to costumise the fonts and even use Windows fonts like Times New Roman or Arial. For ITE thesis it is best to stay with the default settings
\usepackage{lmodern}
\fontfamily{lmr}\selectfont

\usepackage[version=3]{mhchem}


%%%%% Chapter and section settings %%%%%
%% Set list of tables, figures and others in section style
\makeatletter
\renewcommand\listoftables{%
	\subsection*{\listtablename}%
	\@mkboth{\MakeUppercase\listtablename}%
	{\MakeUppercase\listtablename}%
	\@starttoc{lot}%
}
\renewcommand\listoffigures{%
	\subsection*{\listfigurename}%
	\@mkboth{\MakeUppercase\listfigurename}%
	{\MakeUppercase\listfigurename}%
	\@starttoc{lof}%
}
\makeatother

%% Remove chaptercounter for figure and table numbering
\usepackage{chngcntr}
%\counterwithout{figure}{chapter}
%\counterwithout{table}{chapter}

%% Optimises skip between chapter and header for Bachelor and Master Thesis. Can be diabled for Doctoral Thesis.
%\vspace*{2.3\baselineskip} = ORIGINAL 
\renewcommand*{\chapterheadstartvskip}{\vspace*{.5\baselineskip}}% Abstand einstellen

%% Set counters for sections/subsections/subsubsections to be displayed in the toc (table of contents)
%\setcounter{tocdepth}{1}
%\setcounter{secnumdepth}{1}


%%%Colors%%%
\definecolor{ITEgrey}{HTML}{58595b}
\definecolor{TUCgreen}{RGB}{0 140 79}
\definecolor{TUCred}{RGB}{140 28 0}
\definecolor{TUCdarkgrey}{RGB}{128 128 128}
\definecolor{TUClightgrey}{RGB}{230 230 230}
\definecolor{Word}{HTML}{2a579a}


%%%% Others %%%%%
%% Hyperref for references of chapters, sections, tables, figures etc. in text. Should be loaded as last package to not interfere with some other packages. If loaded before glossaries, entries are hyperlinked.
\usepackage[bookmarksnumbered=true]{hyperref}


%%%%% Glossaries and textreferences %%%%%    
\usepackage{fancyref}

\usepackage[nonumberlist,acronym,section=subsection]{glossaries}
\newglossarystyle{myglossarystyle}{%
	\renewenvironment{theglossary}{%
		\begin{labeling}{xxxxxxxxxx}}{\end{labeling}}%
	\renewcommand*{\glossaryentryfield}[5]{%
		\item[\glstarget{##1}{##2}]% Eintragsname:
		\ifthenelse{\equal{##4}{\relax}}{}{\space (##4)}% (Symbol)
		\space ##3% Beschreibung
		%\dotfill ##5% ...Seitenzahl
	}%
}

%% New glossary types to have extra glossaries for units etc.
%\newglossary[slg]{symbols}{sym}{sbl}{Symbols}
%\renewcommand*{\glspostdescription}{}

%\newglossary[un]{units}{uni}{ut}{Units}
%\renewcommand*{\glspostdescription}{}

\makenoidxglossaries
\setglossarystyle{myglossarystyle}
%% In this files are your acronyms and other abbreviations. They have to be refferd to in the document (including the units!) to be listed in the glossaries.
%\loadglsentries{Glossar_symbols.tex}
%\loadglsentries{Glossar_units.tex}
\loadglsentries{Glossar_acronyms.tex}
%\glsdisablehyper %Disables hyperrefed glossary entries

\usepackage{tocstyle}
\newtocstyle[KOMAlike][leaders]{alldotted}{}
\usetocstyle{alldotted}


%%%%% Your data %%%%%
\author{John Johnson}
\newdate{birthday}{01}{01}{1990}
\title{Lorem Ipsum}
%\newdate{hand-in}{01}{01}{2019} %Set the date of your thesis hand in
\newdate{hand-in}{\day}{\month}{\year} %Uses todays date as hand-in date

\bibliography{Literature.bib}

\providecommand{\keywords}[1]
{
	\small	
	\textbf{\textit{Keywords:}} #1
}
\providecommand{\keywordsG}[1]
{
	\small	
	\textbf{\textit{Schlüsselwörter:}} #1
}


\begin{document}
\pagenumbering{Roman}
\begin{titlepage}
\makeatletter
	{\flushleft{\includegraphics[scale=0.7]{sample_logo.png}}\par}
	\centering
	\vspace*{\fill}
	{\large \textbf{Dissertation}\\ by\\ \textbf{John Johnson}\par}
	\vspace*{\fill}
	{\Large \textbf{\@title}\par}
	\vspace*{\fill}
	\begin{centering}
		\begin{table}[!h]
			\centering
			\begin{tabularx}{0.7\textwidth}{@{}l *2{>{\centering\arraybackslash}X}@{}}
				\centering
				1\textsuperscript{st} Supervisor: & \textbf{Prof. Dr. Robert Robertson}\\
				& \\
				2\textsuperscript{nd} Supervisor: & \textbf{Prof. Dr. Tim Timothy}\\
			\end{tabularx}
		\end{table}
	\end{centering}
	\vspace*{\fill}
{\britishdate\displaydate{hand-in}}
\vfill
\flushleft
Institute of Lorem Ipsum\\
Lorem Ipsum University
\makeatother
\end{titlepage}

\thispagestyle{empty}
\vspace*{\fill}
\section*{Declaration of Authorship}
\vspace{15ex}

I have read and understood the guidelines of the Lorem Ipsum University and those published in the Institute of Lorem Ipsum bulletin, including those regarding the use of literature and other sources. 
I confirm that I have prepared this thesis independently by myself. 
Any information taken from other sources and being reproduced in this thesis is clearly referenced.
\vspace{7.5ex}

\noindent
In terms of the general examination regulations, this work has not yet been submitted to any other examination division.
\vspace{7.5ex}

\noindent
I hereby agree that my dissertation may be exhibited in the institute's library and kept for inspection. 

\makeatletter
\vspace{15ex}
\parbox{0.475\textwidth}{\centering City, \britishdate\displaydate{hand-in} \hrule\strut\centering\footnotesize Location, Date} \hfill
\parbox{0.475\textwidth}{\vphantom{City, \britishdate\displaydate{hand-in}} \hrule\strut\centering\footnotesize \@author}
\makeatother
\vfill

\pagestyle{plain}
\include{01Summary}
\include{01Acknowledgements}
\thispagestyle{empty}
\tableofcontents
\newpage
\listoffigures
%\listofschemes
\listoftables	
%\printglossary[title=Glossary]
%\addcontentsline{toc}{subsection}{Glossary}
\printnoidxglossary[type=\acronymtype,title={Abbreviations}]
%\addcontentsline{toc}{subsection}{Abbreviations}
%\printglossary[type=symbols]
%\addcontentsline{toc}{subsection}{Symbols}
%\printglossary[type=units]
%\addcontentsline{toc}{subsection}{Units}
\cleardoublepage
\pagenumbering{arabic}
\pagestyle{new}

\include{10Introduction}
\include{11Chapter1}
\chapter{Motivation}\label{ch:Two}
Lorem ipsum dolor sit amet, consectetur adipiscing elit, sed do eiusmod tempor incididunt ut labore et dolore magna aliqua. Ut enim ad minim veniam, quis nostrud exercitation ullamco laboris nisi ut aliquip ex ea commodo consequat. Duis aute irure dolor in reprehenderit in voluptate velit esse cillum dolore eu fugiat nulla pariatur. Excepteur sint occaecat cupidatat non proident, sunt in culpa qui officia deserunt mollit anim id est laborum.
Figure \ref{fig:figureI} shows a sample image.

\begin{figure}[!htbp]
	\centerline{\includesvg[height=\linewidth]{images/sample}}
	\caption{Sample image.}
	\label{fig:figureI}
\end{figure}

Lorem ipsum dolor sit amet, consectetur adipiscing elit, sed do eiusmod tempor incididunt ut labore et dolore magna aliqua. Ut enim ad minim veniam, quis nostrud exercitation ullamco laboris nisi ut aliquip ex ea commodo consequat. Duis aute irure dolor in reprehenderit in voluptate velit esse cillum dolore eu fugiat nulla pariatur. Excepteur sint occaecat cupidatat non proident, sunt in culpa qui officia deserunt mollit anim id est laborum.
\include{13Chapter3}
\chapter{Main Part}\label{ch:Four}
Lorem ipsum dolor sit amet, consectetur adipiscing elit, sed do eiusmod tempor incididunt ut labore et dolore magna aliqua. Ut enim ad minim veniam, quis nostrud exercitation ullamco laboris nisi ut aliquip ex ea commodo consequat. Duis aute irure dolor in reprehenderit in voluptate velit esse cillum dolore eu fugiat nulla pariatur. Excepteur sint occaecat cupidatat non proident, sunt in culpa qui officia deserunt mollit anim id est laborum.

Lorem ipsum dolor sit amet, consectetur adipiscing elit, sed do eiusmod tempor incididunt ut labore et dolore magna aliqua. Ut enim ad minim veniam, quis nostrud exercitation ullamco laboris nisi ut aliquip ex ea commodo consequat. Duis aute irure dolor in reprehenderit in voluptate velit esse cillum dolore eu fugiat nulla pariatur. Excepteur sint occaecat cupidatat non proident, sunt in culpa qui officia deserunt mollit anim id est laborum.

\begin{equation*}
	\begin{aligned}
		&T\textsubscript{his} = I\textsuperscript{s}\\
		&S\textsubscript{ome} = E\textsuperscript{quation}
	\end{aligned}
\end{equation*}

Lorem ipsum dolor sit amet, consectetur adipiscing elit, sed do eiusmod tempor incididunt ut labore et dolore magna aliqua. Ut enim ad minim veniam, quis nostrud exercitation ullamco laboris nisi ut aliquip ex ea commodo consequat. Duis aute irure dolor in reprehenderit in voluptate velit esse cillum dolore eu fugiat nulla pariatur. Excepteur sint occaecat cupidatat non proident, sunt in culpa qui officia deserunt mollit anim id est laborum.
\bigskip

	\section{Subsection}
	Lorem ipsum dolor sit amet, consectetur adipiscing elit, sed do eiusmod tempor incididunt ut labore et dolore magna aliqua. Ut enim ad minim veniam, quis nostrud exercitation ullamco laboris nisi ut aliquip ex ea commodo consequat. Duis aute irure dolor in reprehenderit in voluptate velit esse cillum dolore eu fugiat nulla pariatur. Excepteur sint occaecat cupidatat non proident, sunt in culpa qui officia deserunt mollit anim id est laborum.
	The listing below shows some sample code.
	
	\pagebreak
	\begin{lstlisting}[language=Python, caption=Sample code]
		while variable == True:
			variable_new = 1
			for part in list:
				do something...	|\Suppressnumber|
	\end{lstlisting}
	
	Lorem ipsum dolor sit amet, consectetur adipiscing elit, sed do eiusmod tempor incididunt ut labore et dolore magna aliqua. Ut enim ad minim veniam, quis nostrud exercitation ullamco laboris nisi ut aliquip ex ea commodo consequat. Duis aute irure dolor in reprehenderit in voluptate velit esse cillum dolore eu fugiat nulla pariatur. Excepteur sint occaecat cupidatat non proident, sunt in culpa qui officia deserunt mollit anim id est laborum.
\chapter{Results}\label{ch:Results}
Lorem ipsum dolor sit amet, consectetur adipiscing elit, sed do eiusmod tempor incididunt ut labore et dolore magna aliqua. Ut enim ad minim veniam, quis nostrud exercitation ullamco laboris nisi ut aliquip ex ea commodo consequat. Duis aute irure dolor in reprehenderit in voluptate velit esse cillum dolore eu fugiat nulla pariatur. Excepteur sint occaecat cupidatat non proident, sunt in culpa qui officia deserunt mollit anim id est laborum.
Table \ref{table:1} shows a sample table.

\begin{table}[h!]
	\centering\noindent
	\renewcommand{\arraystretch}{1.3}
	\begin{tabularx}{\linewidth}{ l >{\raggedleft\arraybackslash}X >{\raggedleft\arraybackslash}X >{\raggedleft\arraybackslash}X >{\raggedleft\arraybackslash}X } 
		\toprule[1.5pt]
		& \multicolumn{4}{c}{This} \\
		\cline{2-5}
		Is & Some & Table & Which & Can \\ 
		\midrule
		Be & 1.00 & 1.00 & 1.00 & 1.00 \\
		Used & 1.00 & 1.00 & 1.00 & 1.00 \\ 
		Advantageously and freely & 1.00 & 1.00 & 1.00 & 1.00 \\ 
		\bottomrule[1.5pt]
	\end{tabularx}
	\caption{Sample table.}
	\label{table:1}
\end{table}

Lorem ipsum dolor sit amet, consectetur adipiscing elit, sed do eiusmod tempor incididunt ut labore et dolore magna aliqua. Ut enim ad minim veniam, quis nostrud exercitation ullamco laboris nisi ut aliquip ex ea commodo consequat. Duis aute irure dolor in reprehenderit in voluptate velit esse cillum dolore eu fugiat nulla pariatur. Excepteur sint occaecat cupidatat non proident, sunt in culpa qui officia deserunt mollit anim id est laborum.
\include{16Chapter6}
\include{21Conclusion}
\newpage
\renewcommand{\bibname}{References}

\printbibliography
\appendix
%% Change of Appendix name
\renewcommand{\appendixtocname}{Appendix}
\addappheadtotoc
\pagestyle{appendix}
\setcounter{table}{0}
\renewcommand{\thetable}{A\arabic{table}}
\setcounter{figure}{0}
\renewcommand{\thefigure}{A\arabic{figure}}
%%%%%%%%
\chapter*{Appendix}
Sample data:
S = [1, 2, 3, 4, 5, 6, 7, 8, 9, 10]
\end{document}


%Tipps und Tricks:
%\section[short title]{title}\label{key} für \nameref{label}, wenn Literaturstelle im Titel wird bei nameref der short title genutzt